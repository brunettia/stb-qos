%!TEX root = ../rapport.tex

\begin{abstract}

A l'heure actuelle et depuis bien longtemps déjà, énormément de projets ont vu le jour afin de rendre l'environnement dans lequel on vit le plus sain et le plus propre possible.

\medskip

Dans le domaine de l'informatique et de la télécommunication, des choses ont été entreprises dans le même but. Il y a par exemple des mouvements comme \emph{"éviter d'imprimer du papier pour rien"}, \emph{"éteindre son ordinateur et son écran losqu'on en a plus besoin"} qui sont encouragés d'une manière ou d'une autre.

\medskip

Certains vont même plus loin en développant des systèmes permettant de détecter si une personne est dans une pièce. On peut par exemple faire en sorte que les lumières s'éteignent automatiquement lorsqu'aucun mouvement n'a été perçu pendant un certain moment. Ceci est, bien entendu, dans le but de faire de l'économie d'énergie.

\medskip

Il existe depuis un certains temps une plateforme informatique qui se nomme \emph{Watt-ICT} permettant de visualiser des informations, interceptées par des capteurs pouvant se positionner entre une prise électrique et des appareils ménagers. Ces données pourraient ensuite être visualisées sous forme graphique par exemple afin d'essayer de rendre les personnes le plus sensible possible.

\medskip

Ce projet de Bachelor, visant également un concept écologique, a comme objectif de reprendre et de modifier la plateforme \emph{Watt-ICT} afin de pouvoir développer, en parrallèle, une application \emph{\gls{ipad}} servant à importer des plans de maisons, positionner des zones dans les pièces désirées et d'y insérer les capteurs disponibles pour la maison. Une fois ceci effectué, l'utilisateur pourra observer facilement les données capturées afin de se faire une idée sur la consommation d'énergie par exemple. 

\end{abstract}



