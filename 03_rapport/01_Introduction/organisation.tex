%!TEX root = ../rapport.tex

\chapter{Organisation} % (fold)
\label{cha:organisation}

Ce chapitre explique l'organisation qui a été tenue tout au long du projet. 

\section{Planification} % (fold)
\label{sec:planification}
La planification est disponible en annexe \ref{cha:annexe:planning}. 

\medskip

Comme la taille du document a du être diminuée pour avoir sa place dans ce rapport, il est possible de trouver le planning au format informatique disponible à l'adresse \cite{online:forge:planfin}.

\medskip

La planification a été découpée en 7 semaines et comporte, dans les grandes lignes, les tâches qui devront être effectuées.

\medskip

Les carrés bleus représentent le temps prévu pour la tâche et les verts sont le temps effectué réellement. Il est ainsi possible de voir par exemple si l'on est trop optimiste ou au contraire trop pessimiste lorsqu'on prévoit des durées pour un projet.

\subsection{Dates clefs} % (fold)
\begin{table}[h] % dates clefs
\begin{tabularx}{\textwidth}{|X|X|X|}
  \hline
  \bf{Activité} & \bf{Jour} & \bf{Heure} \\  \hline
  Début du projet &	Mardi 29.05.2012 & -\\  \hline
Rendu des rapports et du flyer au responsable
de filière	& Vendredi 13.07.2012 &	17:00 \\  \hline
Poster &	Jeudi	06.09.2012 &	12:00 \\  \hline
Présentation des travaux de Bachelor &	Vendredi 07.09.2012 \newline
Samedi	
08.09.2012 &	-\\  \hline
Défense orale &	Lundi 10.09.2012 \newline
Mardi 11.09.2012\newline
Mercredi 12.09.2012 &-\\  \hline
\end{tabularx}
\caption{Tableau récapitualif des dates clefs}
\end{table}
% subsection dates_clefs (end)
% subsection subsection_name (end)
% section planification (end)

\section{Séances} % (fold)
\label{sec:s_ances}
Des séances avec les superviseurs ont été tenues chaque semaine pour expliquer les principales tâches effectuées et ainsi montrer l'avancement du projet. Ces séances permettaient de poser des questions et de recevoir, avec plaisir, des conseils et des feedbacks sur le travail.
% section s_ances (end)

% chapter organisation (end)


\chapter{Environnement de travail} % (fold)
\label{cha:environnement_de_travail}

Le but de ce chapitre est de faciliter la prise en main future du projet par une autre personne. En effet, je reporte ici tous les outils et les appareils que j'ai utilisés pour mener à bien ce projet. Vous trouverez dans ce chapitre également les liens vers les différents documents qui ont été créés durant le travail.

\section{Outils de travail} % (fold)
\label{sec:outils_de_travail}

\subsection{Ordinateur} % (fold)
\label{sub:macbook_pro}
La totalité des activités a été effectuée sur un ordinateur portable d'Apple.

\begin{description}
	\item[Version] MacBook Pro 13 pouces, début 2011
	\item [Système d'exploitation] Mac OS X Lion 10.7.4 
\end{description}
% subsection macbook_pro (end)

\subsection{Tablette} % (fold)
\label{sub:tablette}
L'application tournera sur une tablette de type iPad conçue et développée par Apple.

\begin{description}
	\item[Modèle] iPad 2
	\item [Système d’exploitation] iOS 5.1.1
	\item [Résolution] 1024 x 768 
\end{description}
% subsection tablette (end)

\subsection{Programmes} % (fold)
\label{sub:programmes}
J'énumère ci-dessous les programmes que j'ai utilisés tout au long du projet. Ces programmes sont des versions pour Mac OSX.

\medskip

\begin{itemize}
	\item Sublime Text 2 et TeXLive-2011 pour l'édition du rapport
	\item Visual Paradigm et OmniGraffle version 5.3.6 pour les diagrammes de conception
	\item XCode version 4.3.2 pour la réalisation du client
	\item Eclipse version 3.7.2 pour la réalisation du webservice
	\item WizTools.org RESTClient version 2.4 pour envoyer des requêtes \emph{\gls{rest}} sur un \emph{\gls{webservice}}
	\item MySQL Workbench version 5.2 pour la gestion de la base de données
\end{itemize}
% subsection programmes (end)



% section outils_de_travail (end)

\section{Partage de documents} % (fold)
\label{sec:partage_de_documents}
% section partage_de_documents (end)

Pour partager les documents pendant le projet, j'ai utilisé la \emph{\gls{forge}} mise à disposition par l'\emph{\gls{eiafr}} et disponible à l'adresse \cite{online:forge}.

\medskip

\begin{itemize}
	\item Les versions de la \textbf{planification} sont disponibles à l'adresse \cite{online:forge:plan}
	\item Les versions du \textbf{cahier des charges} sont disponibles à l'adresse \cite{online:forge:spec}
	\item Les \textbf{procès-verbaux} sont disponibles à l'adresse \cite{online:forge:pv}
	\item Ce présent \textbf{rapport} est disponible à l'adresse \cite{online:forge:rapport}. Celui-ci est en deux versions. En effet, il existe une version qui est destinée à être imprimée et une autre destinée à être lue sur un ordinateur. Ceci a été fait dans le but d'avoir un rapport agréable à lire dans les deux cas.
	\item Le \textbf{journal de bord} est disponible à l'adresse \cite{online:forge:wiki}.
\end{itemize}

% chapter environnement_de_travail (end)