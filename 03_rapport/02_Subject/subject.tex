%!TEX root = ../rapport.tex

\chapter{Green Computing}

Ce chapitre a comme objectif d'éclaircir le sujet en définissant de quoi il parle et également de citer quelques concepts de l'\emph{informatique verte}.
\section{Définition} % (fold)
\label{sec:d_finition}
\begin{shadequote}
L'expression anglophone « Green computing » (ou encore « green information technology », en abrégé « green IT ») signifie en français mot à mot « informatique verte », plus largement « informatique éco-responsable ». Le concept désigne un état de l'art informatique qui vise à réduire l'empreinte écologique, économique, et sociale des technologies de l'information et de la communication (TIC). Il s'agit d'une manière globale et cohérente de réduire les nuisances rencontrées dans le domaine des équipements informatiques et ce, « du berceau jusqu'à la tombe » de chaque équipement : soit aux différents stades de fabrication, d'utilisation (consommation d'énergie) et de fin de vie (gestion/récupération des déchets, pollution, épuisement des ressources non renouvelables). \par\emph{Définition tirée de Wikipedia, disponible sur le site \cite{online:wiki:green}}
\end{shadequote}

L'\emph{informatique verte} ou le \emph{Green Computing} ou encore le \emph{Green IT} est un concept visant à réduire la mauvais impact de l'informatique et la télécomunnication sur l'environnement.

\medskip

La définition est tout de fois élargie dans certains cas. Effectivement, on parle également de \emph{Green Computing} dans les cas où, par exemple, des applications sont conçues et développées dans un but écologique. Typiquement, une application qui a pour but de récolter des informations de consommation énergique et d'en faire des graphiques pour les exposer à des personnes peut faire partie du concept de l'\emph{informatique verte}.

\medskip

Ci-dessous, des exemples de concepts faciles à mettre en place et qui font déjà en sorte de rendre le tout \emph{Green} !
% section d_finition (end)


\section{Exemple pour un parc informatique} % (fold)
\label{sec:exemple_pour_un_parc_informatique}

Bien qu'un parc informatique puisse ne pas être \emph{Green}, il est tout à fait possible de le rendre de la sorte.

\medskip

\begin{itemize}
	\item Virtualiser des serveurs et des ordinateurs afin de réduire la consommation électrique. En effet, il est possible de nos jours de rendre virtuelles des machines et ainsi en faire tourner plus d'une sur une seule machine physique.
	\item Supprimer les échanges papier et utiliser les courriers électroniques afin d'économiser le papier
\end{itemize}

\section{Ma contribution pendant le projet} % (fold)
\label{sec:ma_contribution_pendant_le_projet}

Pendant le projet, j'ai essayé de rendre mon train de vie autant écologique que possible ! Je pense que ces petites astuces ont contribué, même si ce n'est qu'un tout petit peu, à l'écologie et au \emph{Green Computing}.

\medskip

\begin{itemize}
	\item Prendre les transports publics pour les longs trajets
	\item Faire les courts trajets à pieds plutôt qu'en véhicule
	\item Eviter d'imprimer beaucoup de documents  et plutôt utiliser les formats électroniques 
	\item Utiliser un ordinateur conçu pour avoir le moins d'impact possible sur l'environnement
\end{itemize}

\begin{shadequote}
MacBook Pro is designed with the following features to reduce environmental impact. \par\emph{Citation tirée du site \cite{online:apple:green}}
\end{shadequote}
% section ma_contribution_pendant_le_projet (end)


% section exemple_pour_un_parc_informatique (end)