%!TEX root = ../rapport.tex

\chapter{Analyse, conception, réalisation et tests de la partie serveur}
\label{cha:analyse_serveur}
Ce chapitre regroupe les différentes parties qui ont été effectuées pour la partie serveur du système. Il s'agit donc d'expliquer le travail qui a été réalisé sur le \emph{\gls{webservice}} .
\section{Analyse} % (fold)
\label{sec:analyse_serveur}
Cette partie décrit l'analyse qui a été faite pour la partie serveur du système. Plus précisément, \emph{\gls{jpa}}, le concept de \emph{\gls{rest}ful} service, \emph{\gls{jersey}} et \emph{\gls{dao}} seront analysés.
\subsection{Java Persistence API} % (fold)
\label{sub:jpa}
\emph{\gls{jpa}} propose une interface qui permet d'intéragir avec une base de données afin de rendre des objets \gls{java} persistants. 

\medskip

\emph{\gls{jpa}} utilise le principe des annotations introduit dans \emph{\gls{java}} 5. Cela permet de définir des classes qui ont pour but de faire l'interface entre la base de données et l'application. C'est un mapping d'objets \emph{\gls{java}} en relation d'une base de données et vice-versa.

\medskip

Le site \cite{online:jpadoudou} fournit une très bonne documentation bien écrite et très complète.

\subsubsection{Annotations}
Ci-dessous, le tableau \ref{tab:jpaannot} met en avant une liste des principales annotations que l'on peut utiliser.

\begin{table}[H]
\begin{tabularx}{\textwidth}{|m{3cm}|X|}
  \hline
  \bf{Annotation} & \bf{Description} \\
  \hline
  @Table & Permet de préciser le nom de la table \\
  \hline  
  @Column & Permet d'associer un champ de la table à une propriété de l'objet \emph{\gls{java}} \\
  \hline  
  @Id & Permet d'associer un champ de la table à une propriété de l'objet \emph{\gls{java}} en tant que clef primaire \\
  \hline
  @GeneratedValue & Permet d'obtenir une clef primaire générée automatiquement \\
  \hline   
\end{tabularx}
\caption{JPA - Liste d'annotations}
\label{tab:jpaannot}
\end{table}

\subsubsection{EntityManager}
Les liens entre la base de données et les entités sont fait par un objet \emph{EntityManager}. Ce dernier peut lire des informations, les modifier, en ajouter ou en supprimer. Il est donc très important lorsqu'on travail avec \emph{\gls{jpa}}

\medskip

Une analyse des différentes opérations a été effectué. Tout d'abord, il faut créer l'\emph{EntityManager} et récuperer la transaction à l'aide du code \ref{lst:entitymanagercreation}. Nous pouvons dès lors effectuer les opérations de type \emph{\gls{crud}}.

\begin{lstlisting}[language={JAVA}, caption={JPA - Création d'une EntityManager}, label={lst:entitymanagercreation}]
Persistence.createEntityManagerFactory("Base");   
EntityManager em = emf.createEntityManager();   
EntityTransaction tr = em.getTransaction();
\end{lstlisting}

Une fois ceci fait, il faut démarrer la transaction et faire un commit quand les opérations ont été effectuées comme le montre le code \ref{lst:transaction}.

\begin{lstlisting}[language={JAVA}, caption={JPA - Gestion de la transaction}, label={lst:transaction}]
transac.begin();
...
transac.commit();
\end{lstlisting}

On peut désormais effectuer les opérations désirées entre le \emph{begin} et le \emph{commit}. Pour ajouter une entité dans la base de donnée, il suffit de faire se procurer l'entité voulue et la persister comme le montre le code \ref{lst:jpa:persist}.

\begin{lstlisting}[language={JAVA}, caption={JPA - Persist}, label={lst:jpa:persist}]
MonObjet monObject = new MonObjet();
monObject.setNom("nom");
em.persist(monObject);
\end{lstlisting}

Pour rechercher dans la base de données, il existe une méthode \emph{find} qui permet de chercher par exemple sur un champ. Il est également possible de le faire par requête directe.

\medskip

Pour supprimer une ocurrence, il est possible d'utiliser la méthode \emph{remove} en lui passant l'entité à éliminer.

\subsubsection{Cascade}
\emph{\gls{jpa}} permet d'effectuer des opérations en cascade. Prenons l'exemple d'une entité \emph{Personne} qui posséderait une entité \emph{Adresse}. On peut alors dire que lorsqu'on supprime une personne dans la base de données, son adresse doit être supprimée automatiquement.

\medskip

Pour illustrer comment cela fonctionnerait avec l'exemple cité ci-dessus, le code \ref{lst:cascaderemove} en montre l'application.

\begin{lstlisting}[language={JAVA}, caption={JPA - Suppression en cascade}, label={lst:cascaderemove}]
@Entity
class Personne {
    @OneToOne(cascade=CascadeType.REMOVE)
    private Adresse sonAdresse;
}
\end{lstlisting}

Il existe plusieurs types de cascade cités dans la liste ci-dessous.

\medskip

\begin{itemize}
  \item PERSIST
  \item MERGE
  \item REMOVE
  \item REFRESH
  \item DETACH
\end{itemize}

\subsection{Service HTTP de type RESTful} % (fold)
\label{sub:service_de_type_restful}

Un service de type \emph{\gls{rest}ful} consiste à pouvoir travailler avec des ressources en utilisant des requêtes du protocole \emph{\gls{http}}. Dans ce protocole, il existe des requêtes comme \emph{GET}, \emph{PUT}, \emph{POST} ou \emph{DELETE}.

\medskip

Le but est donc d'employer ces requêtes en passant des paramètres.

\medskip

Un exemple serait par exemple de vouloir récuperer une information simplement en requêtant l'adresse \texttt{http://service/nom\_de\_la\_ressource}.

% subsection service_de_type_restful (end)
\subsection{Jersey \& JAX-RS} % (fold)
\label{sub:jersey_&_jax_rs}

\emph{\gls{jersey}} est une implémentation de \emph{\gls{jaxrs}} et permet de développer des applications \emph{\gls{java}} qui feront office de \emph{\gls{webservice}} de type \emph{\gls{rest}ful}.

\medskip

Ci-dessous, au code \ref{lst:jaxrs}, se trouve un exemple d'implémentation détaillé par la suite.

\begin{lstlisting}[language={JAVA}, caption={Exemple d'implémentation JAX-RS}, label={lst:jaxrs}]
import javax.ws.rs.*;
import com.sun.jersey.api.json.JSONWithPadding;

@Path("/myRessource")
public class MyRessourceService {

  @Path("/ressources")
  @GET
  @Produces(MediaType.APPLICATION_JSON)
  public List<Ressource> getAllRessources() {
    ...
  }

  @Path("/save")
  @POST
  @Produces(MediaType.APPLICATION_JSON)
  @Consums(MediaType.APPLICATION_JSON)
  public List<Ressource> saveRessource(List<Resssouce> ressource) {
    ...
  }
}
\end{lstlisting}

\subsubsection{Détails du code}
\begin{itemize}
  \item Ce service sera donc accessible via l'url \emph{http://adresse/myRessource}
  \item Pour avor la liste de toute les ressources, il suffira de faire une requête de type \emph{GET} à l'adresse \emph{http://adresse/myRessource/ressources} sans passer de paramètre. La réponse sera au format \emph{\gls{json}}
  \item Pour insérer une ressource dans la base de données, il suffira de faire une requête de type \emph{POST} à l'adresse \emph{http://adresse/myRessource/save} en passant une liste JSON de ressource.
\end{itemize}
% subsection jersey_&_jax_rs (end)

\subsection{Data Access Object}
Les \gls{dao} sont généralement utilisées pour créer des interfaces simplifiant l'accès à une base de données. En effet, au lieu de devoir implémenter tout le processus d'insertion avec l'\emph{EntityManager}, la transaction, etc.. Il suffira d'appeler une méthode en passant les éléments demandés en paramètre.

% section analyse (end)