%!TEX root = ../rapport.tex

\chapter{Conclusion} % (fold)
\label{cha:conclusion}

Ce chapitre conclut ce document et également le projet de Bachelor. Premièrement, un bilan est établit en comparant les activités qui étaient prévues aux travaux qui ont été effectutés réellement. Ensuite, je vous expose mes impressions personnelles sur le projet en détaillant chaques parties du projet. Les problèmes rencontrés durant tout le déroulement du travail sont ensuite reportés sous forme de brèves descriptions. Finalement, je cite les améliorations qui peuvent être faites ou que j'effectuerais si le temps me le permettait.

\section{Bilan} % (fold)
\label{sec:bilan}

Pour effectuer le bilan, j'ai repris chaques tâches du cahier des charges et les ai comparées au travail effectué durant le projet.

\medskip

Pour ce qui est de l'étude du langage, des concepts et de l'environnement de développement \emph{\gls{ios}}, on peut dire que c'est un objectif acquis puisque j'ai vraiment appris énormément de choses et que les concepts maîtrisés m'ont permis de bien avancé durant le projet

\medskip

Pour ce qui est de l'étude et l'analyse des technologies \emph{\gls{http}}, \emph{\gls{json}}, des principes \emph{\gls{rest}} ou \emph{\gls{rest}}ful, \emph{\gls{jaxrs}}, \emph{\gls{jpa}} de la plateforme actuelle, objectif également atteint puisqu'il m'est maintenant très logique et sans trop de difficulté de bien comprendre comment cela fonctionne, d'en tirer les avantages et la puissance de ces technologies.

\medskip

La modification de la plateforme actuelle pour implémenter les besoins de l'application a été effectuée. Il reste cependant des points à détaillés comme par exemple le traîtement des \emph{null} dans les \emph{\gls{dao}}.

\medskip

L'analyse, la conception, l'implémentation et les tests de l'application pour pouvoir positionner des zones ainsi que des capteurs sur le plan et modifier les paramètres de ces derniers a été réalisée. Néanmoins, il reste des petits imperfections, des bugs.

\medskip

Les objectifs secondaires, bien que très intéressants, n'ont pas eu le temps d'etre traîtés.

% section bilan (end)

\section{Impressions personnelles} % (fold)
\label{sec:impression_personnelle}
J'ai sincèrement beaucoup apprécié ce projet autant au niveau technique que méthodologique. 

\medskip

Au niveau des acquis, j'ai vraiment pu approfondir mes connaissances dans le monde de l'informatique. En effet, j'ai pu apprendre pleins de nouvelles choses et de nouveaux concepts dans le développement \emph{\gls{ios}} bien que j'avais déjà pratiqué dans ce domaine. Pour ce qui concerne la mise en place du \emph{\gls{webservice}}, je n'avais jamais utiliser ces technologie auparavant. J'ai donc appris une certaine quantité de principes et de concepts. Cela est d'après moi une très bonne nouvelle et je peux affirmer que je n'hésiterai pas à réutiliser ses technologies pour de futurs projets si cela répond aux besoins.

\medskip

Pour ce qui est de la méthodologie de travail, ce projet m'a une nouvelle fois démontré qu'il était important de faire une bonne analyse afin d'avoir le plus d'outils dans sa boite à outils lors des phases suivantes. Le fait d'avoir fait des petits prototypes lors des phases d'analyse et de conception m'ont également réconforter dans le sens qu'il y avait dans la plupart des cas, quelques choses de visuelles assez rapidement.

\medskip

Malgré le fait que l'application ne soit pas complètement fonctionnelle, je suis dans l'ensemble content du travail fourni et des résultats obtenus lors de ce projet car j'ai pu entraîner et consolider mon expérience lors de projet de développement. Je peux également affirmé que mon souhait de travailler plus tard dans le monde de l'informatique n'a aucunement changé !
% section impression_personnelle (end)

\section{Problèmes rencontrés} % (fold)
\label{sec:probl_mes_rencontr_s}

\subsection{Insertion des capteurs} % (fold)
\label{sub:insertion_des_capteurs}
Actuellement, l'insertion des capteurs ne se passent pas correctement. Si un capteur est déjà dans la base, il n'y pas de soucis, mais on ne peut pas insérer un capteur dans une zone depuis l'application.

\medskip

Le code du coté \emph{\gls{webservice}} fonctionne mais il faut retoucher le code du client.
% subsection insertion_des_capteurs (end)

\subsection{Affichage des zones} % (fold)
\label{sub:affichage_des_zones}
L'affichage des zones ne se fait pas correctement lorsqu'on redimensionne le plan, qu'on sauve et qu'on ouvre à nouveau le même plan. Il faut revoir le code du coté client car j'ai remarqué cela trop tard dans le déroulement du projet.
% subsection affichage_des_zones (end)
% section probl_mes_rencontr_s (end)


\section{Améliorations} % (fold)
\label{sec:am_liorations}
\subsection{Dessiner sa maison} % (fold)
\label{sub:dessiner_sa_maison}
Il serait utile d'avoir un outil permettant de dessiner de notre propre personne le plan de notre maison ou notre appartement dans le cas ou nous n'avons pas un plan existant. Il pourrait se faire d'une manière très simplifié pour commencer en ne dessinant que les murs par exemple. Une application pour \emph{\gls{ios}} existe et permet de faire des plans très détaillés et d'ensuite passer en mode 3d. Il s'agit de \emph{Home Design} qui peut être testé en version gratuite pour se faire une idée.
% subsection dessiner_sa_maison (end)
\subsection{Catégorie d'un capteur} % (fold)
\label{sub:cat_gorie_d_un_capteur}
Un capteur est d'une certaine catégorie. Actuellement, les différents types de catégorie ne sont pas pris en compte. Il serait judicieux d'effectuer une modification afin de par exemple changer l'image du capteur selon sa catégorie.
% subsection cat_gorie_d_un_capteur (end)

\subsection{Actuateur} % (fold)
\label{sub:actuateur}
Le fait de pouvoir insérer des actuateurs pour par exemple éteindre ou allumer une lampe à distance était défini dans les objectifs secondaires. Pour pouvoir utiliser ces actuateurs, il faudrait certainement revoir la conception de la base de données et ensuite adapté le code de l'application cliente.
% subsection actuateur (end)

\subsection{Version iPhone} % (fold)
\label{sub:version_iphone}
On peut désormais assez facilement passer d'une version \emph{\gls{ipad}} à une version \emph{\gls{iphone}}. Pourquoi ne pas proposer deux versions pour les utilisateurs n'ayant pas la chance ou l'envie d'avoir un \emph{\gls{ipad}} et ont uniquement un téléphone mobile ?
% subsection version_iphone (end)


% section am_eliorations (end)

% chapter conclusion (end)

