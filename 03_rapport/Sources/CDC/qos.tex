%!TEX root = ../rapport.tex

\chapter{Quality of Service}

Il existe plusieurs critères qui permettent de savoir si une liaison est de bonne ou mauvaise qualité. Nous allons ici en voir quelques-uns, le but final étant de les intégrer dans notre solution.

\section{Débit}
La mesure du débit nous permet de savoir si celui-ci est bon ou mauvais, s'il est perturbé et s'il est constant.
\section{Jitter}
La gigue, ou jitter en anglais, est la différence entre le temps de variation de  paquets IP successifs sur un certain temps. Les paquets ne mettent pas tous le même temps pour arriver d'un point A à un point B. Ce temps peut varier et la gigue est la différence entre ces variations. Plus elle est petite, mieux c'est.
\section{Perte de paquets}
La perte de paquets IP correspond à une non-délivrance de paquets à destination. Ceci arrive lorsque les buffers des terminaux IP sont pleins. Ceux-ci rejettent alors les paquets arrivant et ils sont perdus.
\section{Round-Trip Time}
Correspond au temps que met un paquet afin de parcourir le circuit entier, aller+retour.

% section structure_du_document (end)