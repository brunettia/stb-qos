%!TEX root = ../rapport.tex

\chapter{Introduction}
Ce chapitre introduit le rapport d'un projet de Bachelor de l'\emph{\gls{eiafr}} en filière informatique. Ce projet s'effectue à la fin des trois ans du cursus scolaire. Ci-dessous, en guise d'introduction, une description du contexte du projet a été repris de la donnée du projet ayant pour but d'expliquer dans les grandes lignes en quoi consiste le travail. Puis les objectifs et leurs tâches relatives sont posés. Pour terminer, vous pourrez trouver comment ce document est structuré.

\section{Contexte} % (fold)
\label{sec:contexte}
Ce projet de Bachelor est mon travail final d’étude à l'\emph{\gls{eiafr}} en section informatique. Ce dernier dure 7 semaines et est cloturé par une défense orale devant les superviseurs, le responsable ainsi que les experts.

\medskip

Vous trouverez ci-dessous l'énoncé original du projet tel qu'il a été présenté aux étudiants lors des choix.

\subsection{Enoncé original du projet} % (fold)
\label{sub:enonc_original_du_projet}
Ce projet porte sur la thématique du \emph{\gls{greecomputing}} 
(informatique écologique). Plus particulièrement, il s’inscrit dans le cadre du projet Hasler Green-Mod portant sur l’analyse de la consommation électrique des appareils ménagers d’un bâtiment en vue de faire de l’économie d’énergie.

\medskip

L'objectif du projet iGreenControl est d'analyser et développer une application \emph{\gls{ipad}} qui permette à des utilisateurs lambda de dessiner le plan de leur maison (ou de l'importer). Une fois que le plan sera acquis, il s’agira de positionner sur le plan les équipements électriques (prises électriques, appareils \emph{\gls{hifi}}, télévisions, ordinateurs, lampes et autres appareils 
ménagers).

\medskip

L'application tournera sur un équipement de type tablette \emph{\gls{ipad}}, Une attention particulière sera portée sur l’ergonomie de l'application. D'un point de vue architecture, le système s'articulera autour d'un serveur permettant la persistance du modèle de données ainsi que le contrôle des équipements. Les contraintes de ce serveur sont connues et un prototype existe déjà : système \emph{\gls{linux}}, base de données \emph{\gls{mysql}}, couche d’accès aux données sur protocole \emph{\gls{http}} avec échange d’informations \emph{\gls{xml}} ou \emph{\gls{json}}.
% subsection enonc_original_du_projet (end)
% section contexte (end)


\section{Objectifs} % (fold)
\label{sec:objectifs}
Cette section décrit les objectifs à atteindre à la fin du projet. Les objectifs secondaires seront réalisés uniquement si les primaires sont terminés et qu'il reste encore du temps au projet.

	\subsection{Primaires}

		\begin{itemize}
		
			\item Analyser et développer une application pour tablette \emph{\gls{ipad}} qui a comme but d'importer le plan d'une maison.
			\item Analyser et implémenter un serveur permettant la persistance du modèle de données.
			\item Permettre à l'utilisateur de positionner des zones sur le plan.
			\item Permettre à l'utilisateur de positionner des capteurs sur le plan et d'en modifier les paramètres.
		\end{itemize}


	\subsection{Secondaires}

		\begin{itemize}
		
			\item Analyser et développer une application pour tablette \emph{\gls{ipad}} qui a comme but de dessiner le plan d'une maison. L'utilisateur pourra également sauver et ouvrir ses plans.
			\item Permettre à l'utilisateur de positionner des actuateurs et de les utiliser.
		
		\end{itemize}
% section objectifs (end)

\section{Tâches} % (fold)
\label{sec:t_ches}
Cette section décrit les activités à effectuer pour atteindre les objectifs à la fin du projet. Les activités secondaires seront réalisées uniquement si les primaires sont terminées et qu'il reste encore du temps au projet.

	\subsection{Primaires}

		\begin{itemize}

			\item L'étude du langage, des concepts et de l'environnement de développement \emph{\gls{ios}}.
			\item L'analyse, la conception, l'implémentation et les tests de l'application pour la tablette iPad pour importer le plan d'une maison.
			\item L'étude et l'analyse des technologies \emph{\gls{http}}, \emph{\gls{json}}, des principes \emph{\gls{rest}} ou \emph{\gls{rest}}ful, \emph{\gls{jaxrs}}, \emph{\gls{jpa}} de la plateforme actuelle.
			\item La modification de la plateforme actuelle pour implémenter les besoins.
			\item L'analyse, la conception, l'implémentation et les tests de l'application pour pouvoir positionner des zones ainsi que des capteurs sur le plan et modifier les paramètres de ces derniers.
		\end{itemize}


	\subsection{Secondaires}

		\begin{itemize}
			\item L'analyse, la conception, l'implémentation et les tests de l'application pour dessiner le plan d'une maison ainsi que pour sauver et ouvrir les plans.
			\item L'analyse, la conception, l'implémentation et les tests de l'application pour pouvoir positionner des actuateurs sur le plan et les utiliser.
		\end{itemize}
% section t_ches (end)

\section{Structure du document} % (fold)
\label{sec:structure_du_document}

Le deuxième chapitre explique l'organisation qui a été tenue tout au long du projet en faisait référence à la planification et aux séances de projet effectueé.

\medskip

Le troisième chapitre a pour but d'aider au maximum la prise en main de travaux futurs sur le projet puisqu'il cite les outils qui ont été employés pour mener à bien le travail.

\medskip

Avec le quatrième chapitre, nous entrons dans le vif du sujet puisqu'il explique les principes, les objectifs et des exemples du \emph{\gls{greecomputing}} à l'heure d'aujourd'hui.

\medskip

Dans le cinquième chapitre, il a été question de faire une analyse de l'application dans sa globalité afin d'entrer dans le projet avec le plus d'informations possibles et répondre aux questions qui restent floues au début du projet. 

\medskip

Le sixième chapitre découle de la conception effectuée pour l'application générale. Cela permet de mieux illustrer les besoins et les objectifs du projet.

\medskip

Le septième chapitre concerne la partie cliente du système. Il s'agit de l'application \emph{\gls{ipad}}. La structure de ce chapitre suit un déroulement assez habituel de développement de logiciel puisqu'il a été découpé en une partie d'analyse, de conception, de réalisation et de tests.

\medskip

Le huitième chapitre comporte la même structure que le septième mais pour la partie serveur du système. Il s'agit donc du \emph{\gls{webservice}} qui sera utilisé par le client.

\medskip 

Dans le chapitre suivant, le neuvième, les tests qui ont été effectués sur l'ensemble du projet ont été reportés.

\medskip

L'avant dernier chapitre, le dixième, intitulé \emph{Résultats}, a pour but d'illustrer par des captures d'écrans les résultats finaux obtenus. Ce chapitre est également un guide pour tous les utilisateurs qui veulent employer l'application.

\medskip

Ce rapport se terminera par une conclusion mettant en avant un bilan, mes impressions personnelles, les problèmes rencontrés ainsi que les améliorations qui peuvent être effectuées.

\medskip

Vous pourrez trouver en annexe le planning effectué au début du projet et la description de la structure du \emph{\gls{dvd}} qui se trouve collé à la fin de ce rapport.

% section structure_du_document (end)