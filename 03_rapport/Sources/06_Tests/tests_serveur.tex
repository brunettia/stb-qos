%!TEX root = ../rapport.tex

\section{Tests}

Pour ce qui est des tests de la partie serveur, j'ai pris chaque requête proposé par l'\emph{\gls{api}} et je l'ai envoyé avec \emph{WizTools.org RESTClient} qui me retourne une réponse bien lisible.
\subsection{Login} % (fold)
\label{sub:login}
Le tableau \ref{tab:stestLogin} reporte les résultats des tests qui ont été effectués pour la partie login de l'application.
\begin{table}[H]
\begin{tabularx}{\textwidth}{|X|X|X|X|}
  \hline
  \bf{Descriptif} & \bf{Paramètres} & \bf{Résultat attendu} & \bf{Résultat obtenu}\\
  \hline
   Se loguer & mot de passe faux & Erreur 404 & \includegraphics[width=16px]{00_media/ok.png} \\
  \hline
     Se loguer & utilisateur faux & Erreur 404 & \includegraphics[width=16px]{00_media/ok.png} \\
  \hline
       Se loguer & informations correctes & \emph{\gls{json}} user & \includegraphics[width=16px]{00_media/ok.png} \\
  \hline
\end{tabularx}
\caption{Tests - Serveur - Résultats des tests pour la partie login}
\label{tab:stestLogin}
\end{table}
% subsection login (end)
\subsection{Maisons} % (fold)
\label{sub:login}
Le tableau \ref{tab:stestMaison} reporte les résultats des tests qui ont été effectués pour la partie maison de l'application.
\begin{table}[H]
\begin{tabularx}{\textwidth}{|X|X|X|X|}
  \hline
  \bf{Descriptif} & \bf{Paramètres} & \bf{Résultat attendu} & \bf{Résultat obtenu}\\
  \hline
  Toutes les maisons & - & Liste des maisons \emph{\gls{json}} & \includegraphics[width=16px]{00_media/ok.png} \\
  \hline
  Les maisons d'un user & id correct & Liste des maisons \emph{\gls{json}} & \includegraphics[width=16px]{00_media/ok.png} \\
  \hline
  Les maisons d'un user & id incorrect & null & \includegraphics[width=16px]{00_media/ok.png} \\
  \hline
   Créer une maison & Maison \emph{\gls{json}} & Maison \emph{\gls{json}} & \includegraphics[width=16px]{00_media/ok.png} \\
  \hline
  Créer une maison & rien & Erreur & \includegraphics[width=16px]{00_media/ok.png} \\
  \hline
   Modifier une maison & rien & Erreur & \includegraphics[width=16px]{00_media/ok.png} \\
  \hline
  Modifier une maison & Modification sur une maison & Maison \emph{\gls{json}} & \includegraphics[width=16px]{00_media/ok.png} \\
  \hline
  Modifier une maison & maison qui n'existe pas & Erreur & \includegraphics[width=16px]{00_media/ok.png} \\
  \hline
  Supprimer une maison & id correct & 204 ok & \includegraphics[width=16px]{00_media/ok.png} \\
  \hline
   Supprimer une maison & id incorrect & 404 & \includegraphics[width=16px]{00_media/ok.png} \\
  \hline
\end{tabularx}
\caption{Tests - Serveur - Résultats des tests pour la partie maison}
\label{tab:stestMaison}
\end{table}

\clearpage

\subsection{Zones} % (fold)
\label{sub:login}
Le tableau \ref{tab:stestZone} reporte les résultats des tests qui ont été effectués pour la partie zone de l'application.
\begin{table}[H]
\begin{tabularx}{\textwidth}{|X|X|X|X|}
  \hline
  \bf{Descriptif} & \bf{Paramètres} & \bf{Résultat attendu} & \bf{Résultat obtenu}\\
  \hline
  Toutes les zones & - & Liste des zones \emph{\gls{json}} & \includegraphics[width=16px]{00_media/ok.png} \\
  \hline
  Les zones d'une maison & id correct & Liste des zones \emph{\gls{json}} & \includegraphics[width=16px]{00_media/ok.png} \\
  \hline
  Les zones d'une maison & id incorrect & null  & \includegraphics[width=16px]{00_media/ok.png} \\
  \hline  
 Créer une zone & zone correcte & zone \emph{\gls{json}} & \includegraphics[width=16px]{00_media/ok.png} \\
  \hline 
  Créer une zone & pas de zone & erreur & \includegraphics[width=16px]{00_media/ok.png} \\
  \hline 
  Créer une zone & zone déjà existante & erreur & \includegraphics[width=16px]{00_media/ok.png} \\
  \hline 
  Modifier une zone & zone déjà existante & zone \emph{\gls{json}} & \includegraphics[width=16px]{00_media/ok.png} \\
  \hline
  Modifier une zone & zone inexistante & zone \emph{\gls{json}} & \includegraphics[width=16px]{00_media/ok.png} \\
  \hline
  Supprimer une zone & zone inexistante & erreur 404  & \includegraphics[width=16px]{00_media/ok.png} \\
  \hline
  Supprimer une zone & zone existante & 204 & \includegraphics[width=16px]{00_media/ok.png} \\
  \hline
\end{tabularx}
\caption{Tests - Serveur - Résultats des tests pour la partie zone}
\label{tab:stestZone}
\end{table}

\clearpage

\subsection{Capteurs} % (fold)
\label{sub:login}
Le tableau \ref{tab:stestSensor} reporte les résultats des tests qui ont été effectués pour la partie capteur de l'application.
\begin{table}[H]
\begin{tabularx}{\textwidth}{|X|X|X|X|}
  \hline
  \bf{Descriptif} & \bf{Paramètres} & \bf{Résultat attendu} & \bf{Résultat obtenu}\\
  \hline
  Toutes les capteurs & - & Liste des capteurs \emph{\gls{json}} & \includegraphics[width=16px]{00_media/ok.png} \\
  \hline
  Les capteurs d'une zone & id correct & Liste des capteurs \emph{\gls{json}} & \includegraphics[width=16px]{00_media/ok.png} \\
  \hline
  Les capteurs d'une zone & id incorrect & null & \includegraphics[width=16px]{00_media/ok.png} \\
  \hline
  Modifier un capteur & - & erreur 404 & \includegraphics[width=16px]{00_media/ok.png} \\
  \hline
  Modifier un capteur & nouveau capteur & null & \includegraphics[width=16px]{00_media/ok.png} \\
  \hline
  Modifier un capteur & Capteur existant & Capteur \emph{\gls{json}} & \includegraphics[width=16px]{00_media/ok.png} \\
  \hline
  Supprimer un capteur d'une zone & Capteur existant & Capteur \emph{\gls{json}} & \includegraphics[width=16px]{00_media/ok.png} \\
  \hline
  Supprimer un capteur d'une zone & Capteur inexistant & null & \includegraphics[width=16px]{00_media/ok.png} \\
  \hline
  Supprimer un capteur d'une zone & Capteur pas dans la zone & Capteur \emph{\gls{json}} & \includegraphics[width=16px]{00_media/ok.png} \\
  \hline
\end{tabularx}
\caption{Tests - Serveur - Résultats des tests pour la partie capteur}
\label{tab:stestSensor}
\end{table}

\clearpage

\subsection{Données du capteur} % (fold)
\label{sub:login}
Le tableau \ref{tab:stestData} reporte les résultats des tests qui ont été effectués pour la partie données du capteur de l'application.
\begin{table}[H]
\begin{tabularx}{\textwidth}{|X|X|X|X|}
  \hline
  \bf{Descriptif} & \bf{Paramètres} & \bf{Résultat attendu} & \bf{Résultat obtenu}\\
  \hline
  Les données pour un capteur & capteur existant & données \emph{\gls{json}} & \includegraphics[width=16px]{00_media/ok.png} \\
  \hline
  Les données pour un capteur & capteur inexistant & null & \includegraphics[width=16px]{00_media/ok.png} \\
  \hline
  Les dernières données pour un capteur & capteur existant & données \emph{\gls{json}} & \includegraphics[width=16px]{00_media/ok.png} \\
  \hline
  Les dernières données pour un capteur & capteur inexistant & null & \includegraphics[width=16px]{00_media/ok.png} \\
  \hline
\end{tabularx}
\caption{Tests - Serveur - Résultats des tests pour la partie données du capteur}
\label{tab:stestData}
\end{table}