%!TEX root = ../rapport.tex


\chapter{Tests sur l'ensemble du projet}
\label{testensembleprojet}
\begin{epigraphs}
\qitem{Program testing can be used to show the presence of bugs, but never to show their absence!}%
{---\textsc{Edsger W. Dijkstra}}
\end{epigraphs}

Ce chapitre reporte les résultats obtenus lors de différents tests effectués sur l'application. Tout d'abord, des tests fonctionnels ont été organisés pour prendre connaissance des éventuels problèmes. Ensuite, des personnes volontaires ont consacré un peu de temps pour "jouer" avec l'application et je reporte ici les principales remarques citées lors de cette expérience.

\section{Tests fonctionnels} % (fold)
\label{sec:tests_fonctionnels}

\subsection{Login} % (fold)
\label{sub:login}
Le tableau \ref{tab:testLogin} reporte les résultats des tests qui ont été effectués pour la partie login de l'application.
\begin{table}[H]
\begin{tabularx}{\textwidth}{|X|X|m{1.5cm}|}
  \hline
  \bf{Descriptif} & \bf{Résultat attendu} & \bf{Résultat obtenu} \\
  \hline
  Le serveur n'est pas atteignable et l'utilisateur se loggue & Message d'information & \includegraphics[width=16px]{00_media/ok.png} \\
  \hline
  Le serveur est atteignable et l'utilisateur se loggue avec des paramètres non valides & Message d'information & \includegraphics[width=16px]{00_media/ok.png} \\
  \hline
    Le serveur est atteignable et l'utilisateur se loggue avec des paramètres valides & L'application change de vue & \includegraphics[width=16px]{00_media/ok.png} \\
  \hline
\end{tabularx}
\caption{Tests - Résultats des tests pour la partie login}
\label{tab:testLogin}
\end{table}
% subsection authentification (end)

\subsection{Logout} % (fold)
\label{sub:logout}
Le tableau \ref{tab:testLogout} reporte les résultats des tests qui ont été effectués pour la partie logout de l'application.
\begin{table}[H]
\begin{tabularx}{\textwidth}{|X|X|m{1.5cm}|}
  \hline
  \bf{Descriptif} & \bf{Résultat attendu} & \bf{Résultat obtenu} \\
  \hline
  Le serveur n'est pas atteignable et l'utilisateur se déloggue & L'application change de vue & \includegraphics[width=16px]{00_media/ok.png} \\
  \hline
  Le serveur est atteignable et l'utilisateur se déloggue  & L'application change de vue & \includegraphics[width=16px]{00_media/ok.png} \\
  \hline
\end{tabularx}
\caption{Tests - Résultats des tests pour la partie logout}
\label{tab:testLogout}
\end{table}
% subsection logout (end)

\subsection{Liste des maisons} % (fold)
\label{sub:logout}
Le tableau \ref{tab:testListMaison} reporte les résultats des tests qui ont été effectués lors de l'affichage de la liste des maisons.
\begin{table}[H]
\begin{tabularx}{\textwidth}{|X|X|m{1.5cm}|}
  \hline
  \bf{Descriptif} & \bf{Résultat attendu} & \bf{Résultat obtenu} \\
  \hline
  L'utilisateur affiche la liste des maisons&La liste s'affiche correctement & \includegraphics[width=16px]{00_media/ok.png} \\
  \hline
  L'utilisateur affiche la liste des maisons et le serveur n'est pas accessible & La liste s'affiche correctement & \includegraphics[width=16px]{00_media/ok.png} \\
\hline
  L'utilisateur filtre les résultats en utilisant le champ de recherche & La liste s'affiche et prend en compte l'entrée de l'utilisateur & \includegraphics[width=16px]{00_media/ok.png} \\
  \hline
    L'utilisateur filtre les résultats en utilisant le champ de recherche et le serveur n'est pas accessible & La liste s'affiche et prend en compte l'entrée de l'utilisateur & \includegraphics[width=16px]{00_media/ok.png} \\
  \hline
    L'utilisateur clique sur une maison & La maison s'affiche correctement & \includegraphics[width=16px]{00_media/ok.png} \\
  \hline
      L'utilisateur clique sur une maison et le serveur n'est pas accessible & Message d'informations concernant le problème & \includegraphics[width=16px]{00_media/ok.png} \\
  \hline
\end{tabularx}
\caption{Tests - Résultats des tests pour la partie liste de maisons}
\label{tab:testListMaison}
\end{table}
% subsection logout (end)

\clearpage

\subsection{Affichage du plan} % (fold)
\label{sub:logout}
Le tableau \ref{tab:testAffichagePlan} reporte les résultats des tests qui ont été effectués lors de l'affichage du plan.
\begin{table}[H]
\begin{tabularx}{\textwidth}{|X|X|m{1.5cm}|}
  \hline
  \bf{Descriptif} & \bf{Résultat attendu} & \bf{Résultat obtenu} \\
  \hline
  Le plan existe dans dans la librairie & L'image s'affiche & \includegraphics[width=16px]{00_media/ok.png} \\
  \hline
  Le plan n'existe  pas dans la librairie & Message d'information & \includegraphics[width=16px]{00_media/ok.png} \\
  \hline

\end{tabularx}
\caption{Tests - Résultats des tests pour la partie de l'affichage du plan}
\label{tab:testAffichagePlan}
\end{table}
% subsection logout (end)

\subsection{Affichage des zones} % (fold)
\label{sub:logout}
Le tableau \ref{tab:testAffichageZones} reporte les résultats des tests qui ont été effectués lors de l'affichage des zones.
\begin{table}[H]
\begin{tabularx}{\textwidth}{|X|X|m{1.5cm}|}
  \hline
  \bf{Descriptif} & \bf{Résultat attendu} & \bf{Résultat obtenu} \\
  \hline
  La maison comporte une ou des zones & La ou les zones sont affichées sur le plan & \includegraphics[width=16px]{00_media/ok.png} \\
  \hline
  La maison ne comporte aucune zone & Le plan est affiché normalement sans zones & \includegraphics[width=16px]{00_media/ok.png} \\
  \hline
  La maison comporte des zones qui n'ont pas de paramètres d'affichage (taille, position)& Les zones sont affichées tout en haut à gauche du plan & \includegraphics[width=16px]{00_media/ok.png} \\
  \hline
\end{tabularx}
\caption{Tests - Résultats des tests pour la partie de l'affichage des zones}
\label{tab:testAffichageZones}
\end{table}

\subsection{Modification des zones} % (fold)
\label{sub:logout}
Le tableau \ref{tab:testAffichageZones} reporte les résultats des tests qui ont été effectués lors du déplacement, l'agrandissement et la rotation des zones sur le plan.
\begin{table}[H]
\begin{tabularx}{\textwidth}{|X|X|m{1.5cm}|}
  \hline
  \bf{Descriptif} & \bf{Résultat attendu} & \bf{Résultat obtenu} \\
  \hline
  L'utilisateur déplace une zone  & La zone se déplace mais ne sort pas du plan & \includegraphics[width=16px]{00_media/ok.png} \\
  \hline
  L'utilisateur modifie la taille une zone  & La zone agrandit ou diminue sa taille  & \includegraphics[width=16px]{00_media/ok.png} \\
  \hline
  L'utilisateur tourne une zone  & La zone se tourne  & \includegraphics[width=16px]{00_media/ok.png} \\
  \hline
  L'utilisateur déplace une zone après l'avoir agrandi & Mouvement normal & \includegraphics[width=16px]{00_media/not_ok.png} \\
  \hline
\end{tabularx}
\caption{Tests - Résultats des tests pour la partie de la modification des zones}
\label{tab:testAffichageZones}
\end{table}
Il y a effectivement encore un bug avec le mouvement ou la rotation des zones après un agrandissement. 
% subsection logout (end)
% section tests_fonctionnels (end)

\section{Expériences utilisateurs} % (fold)
\label{sec:exp_riences_utilisateurs}
J'ai à plusieurs reprises demandé à des connaissances de tester mon application. Ce qu'il en est principalement ressorti est un point que j'avais déjà soulevé concernant le bug de déplacement des zones après un agrandissement.

\medskip

Il y aussi le fait que certaines personnes ne savaient pas trop quand il fallait enregistrer, si cela était fait automatiquement, etc ..

\medskip

D'une manière général, ils ont compris assez vite le fonctionnement de l'application et ont même trouvé cela assez intuitif.

\medskip

Au niveau du design, il est ressorti que les couleurs étaient bien choisies car elles mettaient en confiance l'utilisateur.
% section exp_riences_utilisateurs (end)