%!TEX root = ../rapport.tex

\section{Réalisation}
Cette section a pour but de présenter certains détails jugés importants par rapport à l'implémentation.

\medskip

Pour se faire, je vais expliqué chaque package qui mérite explications et qui a été modifiée pendant le projet. Voici pour commencer la liste des packages en question: 

\medskip

\begin{itemize}
	\item \emph{\gls{dao}} qui est le package d'interface vers la base de données
	\item \emph{\gls{dto}} qui permet de convertir un objet dans un format demandé par l'utilisateur
	\item \emph{Mapper} qui retourne des objets \emph{\gls{java}} correspondant aux entités \emph{\gls{jpa}} passées en paramètres
	\item \emph{Model} qui contient les entités \emph{\gls{jpa}}
	\item \emph{Resource} qui contient les ressources implémentées en \emph{\gls{jersey}}
\end{itemize}

\subsection{HouseDAO} % (fold)
\label{sub:housedao}
Cette classe contient donc des méthodes allant interroger la base de données. Typiquement, le code \ref{lst:serverdao} sélectionne la maison avec un certain id.
\begin{lstlisting}[language={JAVA}, caption={DAO - Maison avec id}, label={lst:serverdao}]
public Tbl_House getHouseById(int id) {
	
	EntityManager em = DBManager.getInstance().getNewEntityManager();
	try {
		TypedQuery<Tbl_House> q = em.createNamedQuery("House.findById", Tbl_House.class);
		q.setParameter("id", Long.valueOf(id));
		return q.getSingleResult();
	} finally {
		em.close();
	}

}
\end{lstlisting}

On utilise ci-dessus une requête nommée qui a été inséré directement dans la classe de l'entité concerné sous la forme du code \ref{lst:namedrequest}.

\begin{lstlisting}[language={JAVA}, caption={Requête nommée}, label={lst:namedrequest}]
@NamedQuery(name="House.findById", query="SELECT t FROM Tbl_House t where t.ho_PK_House = :id ")
\end{lstlisting}
% subsection housedao (end)