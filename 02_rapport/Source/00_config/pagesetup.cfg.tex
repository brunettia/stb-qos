%!TEX root = ../rapport.tex

\hypersetup{
  pdfauthor = {\authornameone},
  pdftitle = {\reportpdftitle},
  pdfsubject = {\modulename, \documenttitle},
  unicode=false,		  % non-Latin characters in Acrobat’s bookmarks
  pdftoolbar=true,		  % show Acrobat’s toolbar?
  pdfmenubar=true,		  % show Acrobat’s menu?
  pdffitwindow=false,	  % window fit to page when opened
  pdfstartview={FitH},	  % fits the width of the page to the window
  pdfcreator={LaTeX},	  % creator of the document
  pdfproducer={latexpdf}, % producer of the document
  pdfkeywords={Report},	  % list of keywords
  pdfnewwindow=true,	  % links in new window
  colorlinks=true,		  % false: boxed links; true: colored links
  linkcolor=black,		  % color of internal links
  citecolor=black,		  % color of links to bibliography
  filecolor=blue,		  % color of file links
  urlcolor=black,		  % color of external links
  linktoc=all
}

\definecolor{gray}{rgb}{0.4,0.4,0.4}
\definecolor{darkblue}{rgb}{0.0,0.0,0.6}
\definecolor{cyan}{rgb}{0.0,0.6,0.6}

\setmargins{2.5cm}%		% left edge
		   {1.5cm}%		% top edge
		   {15.5cm}%	% text width
		   {23.42cm}%	% text hight
		   {14pt}%		% header hight
		   {1cm}%		% header distance
		   {0pt}%		% footer hight
		   {2cm}%		% footer distance

\lstset{
	basicstyle=\footnotesize\ttfamily, % Standardschrift
	numbers=left,				% Ort der Zeilennummern
	numberstyle=\tiny,			% Stil der Zeilennummern
	stepnumber=1,				% Abstand zwischen den Zeilennummern
	numbersep=5pt,				% Abstand der Nummern zum Text
	tabsize=2,					% Groesse von Tabs
	extendedchars=true,			%
	breaklines=true,			% Zeilen werden Umgebrochen
	keywordstyle=\color{blue},
	frame=b,		
	keywordstyle=[1]\textbf,	% Stil der Keywords
	keywordstyle=[2]\textbf,	%
	keywordstyle=[3]\textbf,	%
	keywordstyle=[4]\textbf	  %\sqrt{\sqrt{}} %
	stringstyle=\color{blue}\ttfamily, % Farbe der String
	showspaces=false,			% Leerzeichen anzeigen ?
	showtabs=false,				% Tabs anzeigen ?
	xleftmargin=17pt,
	framexleftmargin=17pt,
	framexrightmargin=0pt,
	framexbottommargin=4pt,
	backgroundcolor=\color{listinglightgray},
	showstringspaces=false,		 % Leerzeichen in Strings anzeigen ?	
	commentstyle=\color{listinggreen}
}
\lstset{
language=Java,
basicstyle=\normalsize, % ou ça==> basicstyle=\scriptsize,
upquote=true,
aboveskip={1.5\baselineskip},
columns=fullflexible,
showstringspaces=false,
extendedchars=true,
breaklines=true,
showtabs=false,
showspaces=false,
showstringspaces=false,
identifierstyle=\ttfamily,
keywordstyle=\color[rgb]{0,0,1},
commentstyle=\color[rgb]{0.133,0.545,0.133},
stringstyle=\color[rgb]{0.627,0.126,0.941},
}
\lstdefinelanguage{XML}
{
  morestring=[b]",
  morestring=[s]{>}{<},
  morecomment=[s]{<?}{?>},
  stringstyle=\color{black},
  identifierstyle=\color{darkblue},
  keywordstyle=\color{cyan},
  morekeywords={xmlns,version,type}% list your attributes here
}
 \lstloadlanguages{% Check Dokumentation for further languages ...
         %[Visual]Basic
         %Pascal
         C,
         C++,
         XML,
         HTML,
         Java
 }

\DeclareCaptionFont{white}{\color{white}}
\DeclareCaptionFormat{listing}{\colorbox{listinggray}{\parbox{\dimexpr\textwidth-2\fboxsep\relax}{#1#2#3}}}
\captionsetup[lstlisting]{format=listing,labelfont=white,textfont=white, singlelinecheck=false, margin=0pt, font={bf,footnotesize}}

\lstdefinelanguage{CSS} 
{morekeywords={color,background,margin,padding,font,weight,display,position,top,left,right,bottom,list,style,border,size,white,space,min,width},
sensitive=false, 
morecomment=[l]{//}, 
morecomment=[s]{/*}{*/}, 
morestring=[b]", 
} 
 

