%!TEX root = ../rapport.tex

\newglossaryentry{ipad}{
    name = {iPad}, 
    description = {Tablette tactile conçue et développée par Apple},
}

\newglossaryentry{iphone}{
    name = {iPhone}, 
    description = {Téléphone mobile tactile conçue et développée par Apple},
}
\newglossaryentry{java}{
    name = {Java},
    description = {Langage de programmation orienté objet}
}

\newglossaryentry{eclipse}{
    name = {Eclipse},
    description = {Environnement de développement libre, extensible et polyvalent de la Fondation Eclipse}
}

\newglossaryentry{obj-c}{
    name = {Objective-C},
    description = {Langage de programmation orienté objet}
}

\newglossaryentry{xcode}{
    name = {XCode},
    description = {Environnement de développement d'Apple permettant de créer des applications pour Mac, iPhone et iPad}
}

\newglossaryentry{javascript}{
    name = {JavaScript},
    description = {Langage de programmation de script essentiellement utilisé dans le monde web}
}

\newglossaryentry{webservice}{
    name = {WebService},
    description = {Interface d'échange d'informations, de données qui s'effectue à travers le web},
    plural=WebServices
}

\newglossaryentry{mysql}{
    name = {MySQL},
    description = {MySQL est un système de gestion de base de données},
}

\newglossaryentry{ios}{
    name = {iOS},
    description = {Système d'exploitation développé par Apple},
}

\newglossaryentry{greecomputing}{
    name = {Green Computing}, 
    description = {Principe qui consiste à améliorer l'environnement dans le domaine de l'informatique}   ,
}

\newglossaryentry{linux}{
    name = {Linux}, 
    description = {Linux ou GNU/Linux, est un système d'exploitation libre},
}

\newglossaryentry{jersey}{
    name = {Jersey},
    description = {Jersey permet d'implémenter des webservices de type REST en java},
}

\newglossaryentry{maven}{
    name = {Maven}, 
    description = {Outil permettant d'automatiser le déploiement d'application Java},
}

\newglossaryentry{singleton}{
    name = {Singleton}, 
    description = {Un Singleton est un terme utilisé pour designer une classe qui est instancié une seule et unique fois. Il s'agit également d'un pattern de conception},
}


\newglossaryentry{forge}{
    name = {Forge}, 
    description = {Plateforme collaborative de gestion de projets de l'institut des Technologies de l'information et de la communication}
}