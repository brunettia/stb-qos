%!TEX root = ../rapport.tex

\section{Github}
GitHub est un site web permettant l'hébergement et la gestion de logiciel. Il utilise le gestionnaire de version Git.

Mes projets sont donc disponibles sur GitHub à l'adresse suivante:


\smallskip


\url{https://github.com/brunettia/stb-qos}

\medskip

On retrouve ici tout ce qui concerne mon projet:
\smallskip

\begin{itemize}
	\item {\bf 01\_cahiers\_des\_charges}: Nous retrouvons ici les différentes versions du cahier des charges, ainsi que le planning du projet.
	\item {\bf 02\_pv}: Tous les procès verbaux des séances que j'ai eu durant le projet se retrouvent ici.
	\item {\bf 03\_rapport}: Le rapport du projet ainsi que ses documents annexes sont disponibles dans ce dossier.
	\item {\bf 04\_android\_app}: L'application Android de la Set-Top Box avec son code source.
	\item {\bf 05\_server\_app}: La partie WebSockets et Web Services du serveur est hébergée ici.
\end{itemize}

\medskip

Pourquoi avoir utilisé GitHub ? Les projets hébergés sur GitHub, à moins qu'ils soient privés, sont disponibles pour tout le monde. Il est possible de lire les sources directement en ligne ainsi que de "forker" le projet. Cela veut dire que n'importe qui peut le modifier.

\medskip

Après discussion avec Olivier, il n'y avait pas de problème à ce que le projet soit Open Source, du moment que les informations relatives à l'authentification ne soient pas disponibles au grand public.

\medskip

L'avantage à présent c'est que le projet est disponible et visible, et peut donc aider les utilisateurs traitant du même sujet que le mien. J'ai moi-même beaucoup utilisé GitHub pour voir certaines conceptions et des exemples de développement.